\documentclass[12pt,letterpaper]{article}
\usepackage[left=1in,right=1in,top=1in,bottom=1in]{geometry}
\usepackage{amssymb}
\usepackage[fleqn]{amsmath}
\usepackage{amsthm}
\usepackage{graphicx}
\usepackage{listings}
\usepackage{courier}
\usepackage[usenames,dvipsnames]{color}    
\DeclareGraphicsExtensions{.png,.pdf}
\graphicspath{}

\title{The Influence of Receiver Routes and Separation from Defenders on Third Down Conversions}
\author{Lingge Li, Hua Xie, Micah Jackson, and Kyle Sneeden}

\linespread{1.25}

\begin{document}
\lstset{ 
  language=R,                     % the language of the code
  basicstyle=\tiny\ttfamily, % the size of the fonts that are used for the code
  numbers=left,                   % where to put the line-numbers
  numberstyle=\tiny\color{Blue},  % the style that is used for the line-numbers
  stepnumber=1,                   % the step between two line-numbers. If it is 1, each line
                                  % will be numbered
  numbersep=5pt,                  % how far the line-numbers are from the code
  backgroundcolor=\color{white},  % choose the background color. You must add \usepackage{color}
  showspaces=false,               % show spaces adding particular underscores
  showstringspaces=false,         % underline spaces within strings
  showtabs=false,                 % show tabs within strings adding particular underscores
  frame=single,                   % adds a frame around the code
  rulecolor=\color{black},        % if not set, the frame-color may be changed on line-breaks within not-black text (e.g. commens (green here))
  tabsize=2,                      % sets default tabsize to 2 spaces
  captionpos=b,                   % sets the caption-position to bottom
  breaklines=true,                % sets automatic line breaking
  breakatwhitespace=false,        % sets if automatic breaks should only happen at whitespace
  keywordstyle=\ttfamily,      % keyword style
  commentstyle=\color{YellowGreen},   % comment style
  stringstyle=\color{ForestGreen}      % string literal style
}

\maketitle

\section*{Introduction}

As the New England Patriots demonstrated in their recent victory over the Kansas City Chiefs in the AFC Championship on January 20th, 2019, repeated third and long conversions can demoralize and exhaust a defense into submission. It goes without saying that maintaining a robust rate of third down conversions is integral to a consistently productive offense at any level of football. Should a head coach and offensive coordinator opt for a passing play on a third down (perhaps on a third and long situation), what should they keep in mind with respect to receiver routes and movements to increase their chances of conversion? To help answer this question, we will survey features of successful third down conversions in the data set covering 91 games from the first six weeks of the 2017 NFL season in our report.

In particular, to constrain the scope of this report and to fit it within the domain of Theme 3, receivers and routes, we analyzed third down forward pass plays that occurred in the data set. For simplification purposes, we define third down success as subsequent advancement to a first down without looking at third downs that led to fourth down plays. There were 1,734 third down pass plays in total, of which 1,030 resulted in failure.

Over the course of our analysis, we first investigated our initial hypothesis that successful third down conversions demonstrate larger separation by yardage between eligible receivers and defenders at the point in time right before the execution of the forward pass. We then classified third down receiver routes at a fundamental level using information such as receiver movement angle and distance traveled downfield calculated from play data. Lastly, we built and analyzed logistic statistical models using route features as predictors to analyze whether certain features, such as basic route type or distance to first marked route angle change, were significant in predicting separation or successful conversion. Our results are detailed on the following pages.

\section*{Receiver Separation in Third Down Pass Plays}

Prior to digging into route features, we first sought to identify cursory quantitative differences between the sample of third down pass-forward plays that failed or that which succeeded in an exploratory analysis. For each play, we isolated the frame immediately preceding the pass, by which point the QB has made his post-snap reads and has made his throwing decision. At this frame, we calculated the minimum distance of each ``standard'' eligible receiver in the play from his closest defender. 

We determined standard eligible receivers by identifying non-QB players whose numbers were under 50 or greater than 79. Because plays in which offensive linemen and defensive players reported as eligible were inconsistently labeled in the data, we did not look at the paths of ``non-standard'' eligible personnel. Trick plays in which the QB was the receiver were not considered. Consequently, we obtained a vector of up to five non-zero distances between eligible receivers and their closest defender for each qualified third down passing play.       

After iterating through all qualified third down passing plays, we plotted the means and maxes of these vectors in two violin plots.

Logistic model.

Ah, you are regressing to a different time window.

Two figures -- max of min and mean of min

What about hypothesis?

What percent of third down conversions were passing versus rushes? We look to see if we can identify what corresponds to successful third down conversions. Third conversions with forward pass

Both mean of mins and max of mins separations show that successful third down forward pass conversions had slightly higher separations.

Third down conversions are 

In

First six weeks of games

\section*{Logistic Regression Models on Third Down Pass Plays}

Basic route type determined from the angle change at break point

Odds ratio.

If given more time, we would have also liked to have maximum separation between a receiver and a defender as a factor in the model.

Algorithimically classifying routes can be somewhat noisy?

Two figures

\section*{Example Plays}

Positive Example -- 2505 (successful post)

Negative Example -- Curl (Tom Brady negative route, 2 yards short, hard to change direction to to 1st down)

Positive Example -- Good separation

Three figures

To illustrate the effect of separation and routes at a qualitative level, we look at two particular plays.

\section*{Conclusion}

Data can also be applied to pass plays on other downs. Future analysis, etc. Next step unsupervised ML classification of route combinations.

\end{document}
